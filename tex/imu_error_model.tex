\documentclass{article}

\usepackage{amsmath, amsfonts}
\usepackage[colorlinks]{hyperref}

\usepackage[format=plain,
            labelfont={bf,it},
            textfont=it]{caption}

\usepackage{graphicx, subcaption}
\graphicspath{ {./images/} }

%\usepackage{biblatex} %Imports biblatex package
\usepackage[backend=bibtex]{biblatex}
%\usepackage[backend=biber]{biblatex}


\addbibresource{./strapdown_report.bib} %Import the bibliography file

%\usepackage{biblatex} %Imports biblatex package
%\usepackage[backend=bibtex,style=alphabetic]{biblatex}
%\usepackage[toc]{appendix}  % Show appendices in table of contents.

%\addbibresource{my_library.bib} %Import the bibliography file

\title{Inertial sensor unit error model}
\author{Henk Luinge  \\
	infertia.com   \\
	}
	
\date{\today}

\begin{document}

\maketitle

\begin{abstract}
We describe how the uncertainty of the IMU orientation and velocity estimation propagates over time, given a baseline random walk model.
\end{abstract}

\section{Introduction}

This report deals with the derivation: Given an orientation with some uncertainty as well as an uncertaint angular velocity, what is the orientation and it's uncertainty ater an integration step?  The angular velocity has an uncertain baseline uncertain baseline. We assume all uncertainties are gaussian. 

The orientation propagation can be used as part of a larger maximum likelihood mechanization such as a Kalman filter. We assume consumer-grade inertial sensors as decribed in the strapdown navigation report \cite{Luinge2025}.

\section{Background and definitions}

\paragraph{Small angle rotation}
We use the helical angle approximation as described in  \cite{Bortz1970} to approximate a rotation from A to B over a small angle $\Delta \varphi$.

\begin{equation} \label{eq:small_angles_orientation_update}
{}^{SB} \varphi = {}^{SA} \varphi + \Delta \varphi + {}^{SA} \varphi \times \Delta \varphi
\end{equation}
This equation is only valid for small angles $\varphi $ as well as $\Delta \varphi$.

The coordinate systems are the same as defined in \cite{Luinge2025}: L stands for the local, earth-fixed coordinate frame and S for the imu frame.

\paragraph{Orientation error}
The orientation error is described using the axis-angle representation $\theta$ and the baseline $b_t$, both 3 element vectors.

\begin{equation} \label{eq:orientation_error_definition}
{}^{LS} \hat{R} = {}^{LS} R \cdot R_{\epsilon} \approx {}^{LS} R \cdot \left[ I + \theta\times  \right]
\end{equation}

The capital $I$ is the 3x3 identity matrix. $\theta$ is the axis and angle to rotate from the real to the estimated orientation, starting from the actual (but unknown) S frame. In typical Kalman filtering applications, the unknown $\theta$ its described using as a zero-mean gaussian probability density function (pdf).

\begin{equation} \label{eq:angular_velocity_model}
y_{gyr,t} =  \hat{\omega}_t = \omega_t + b_t
\end{equation}

\section{Error propagation}
Using \ref{eq:small_angles_orientation_update} and starting from an error axis-angle $\varphi$:
\begin{equation} \label{eq:phi_propagation}
\varphi_t =  \varphi_{t-1} + T\omega + \varphi_{t-1} \times T\omega 
\end{equation}

\subsection{Strapdown integration}
Assuming constant angular velocity during a sampling interval, the change in orientation between two timesteps is computed as:

\begin{equation} \label{eq:strapdown_orientation_including_bias}
\Delta \hat{R}_t \equiv {}^{S_{t-1}S_t} \hat{R} \approx \left[ I + \omega \times   +  \hat{b}\times  \right]
\end{equation}

or including the previous orientation:
\begin{equation} \label{eq:strapdown_orientation_propagation_rotation_matrix}
\begin{aligned}
{}^{LS} \hat{R}_t &= {}^{LS} \hat{R}_{t-1} \cdot  \Delta \hat{R}_t  \\
&= {}^{LS} R_{t-1} \cdot \left[ I + \theta\times  \right] \cdot \left[ I + \omega \times   +  \hat{b}\times  \right] 
\end{aligned}
\end{equation}

\subsection{Acceleration error model}
When the accelerometer is used to compute the acceleration, the gravitational component needs to be added accounted for:
\begin{equation} \label{eq:accelerometer_signal}
\boldsymbol{y}_{Acc} = {}^S\boldsymbol{a} - {}^S\boldsymbol{g}
\end{equation}


Using the accelerometer relation, we compute the acceleration in the local frame as:

\begin{equation} \label{eq:acceleration_estimate}
\begin{aligned}
{}^L\hat{\boldsymbol{a}} &= {}^{LS}\hat{\boldsymbol{R}} \cdot \boldsymbol{y}_{Acc} + {}^L\boldsymbol{g} \\
&= {}^{LS}\boldsymbol{R} \cdot \left[ I + \theta\times  \right] \cdot \boldsymbol{y}_{Acc} + {}^L\boldsymbol{g} \\
&= {}^L\boldsymbol{a} + \left[ \boldsymbol{\theta}\times  \right] \boldsymbol{y}_{Acc} \\
&= {}^L\boldsymbol{a} - \left[ \boldsymbol{y}_{Acc} \times  \right] \boldsymbol{\theta}
\end{aligned}
\end{equation}

\subsection{Angular velocity}
The angular velocity is measured in the IMU coordinate frame $S$. The estimate in the local frame depends on the angle error:

\begin{equation} \label{eq:angular_velocity_estimate}
\begin{aligned}
{}^{L}\hat{\omega} &= {}^{LS}\hat{\boldsymbol{R}} \cdot {}^{S}\hat{\omega}  \\
&= {}^{LS}\boldsymbol{R} \cdot \left[ I + \theta\times  \right] \cdot {}^{S}\hat{\omega}  \\
&= {}^{L}{\omega} + {}^{S}\hat{\omega} \times \theta
\end{aligned}
\end{equation}


\section{Joints}
We'll separate joint constraints in translation constraints and rotation constraints. The translation constraints are common with ball-and-socket joints and hinge joints. The rotation contrains further specifies limits uin rotation such as a hinge constraint.

\subsection{Tranlation constraints}
If the position/acceleration/velocity of a joint is known, an informative constraint can be formulated. For example if it is known that the velocity of the joint is $\boldsymbol{v}_{joint}$.

\begin{equation} \label{eq:joint_velocity_constraint}
\boldsymbol{v}_{Joint} = \boldsymbol{v}_{IMU} - \boldsymbol{\omega} \times \boldsymbol{r}
\end{equation}

\subsection{Rotation constraints}
In case a joint can rotate around a single axis $n$, another constraint can be formulated:

\begin{equation} \label{eq:joint_rotation_constraint}
\boldsymbol{0} = \boldsymbol{n}_{joint} \cdot \boldsymbol{\omega} 
\end{equation}
The equation becomes interestng in case the joint direction $n$ is formulated in the local coordinates instead of teh segment coordiantes. Separating known terms to the left and error terms on the right of the is equal sign. The error term on the right is a linear relation of the orientation error $\theta$.

\begin{equation} \label{eq:joint_velocity_constraint_linearized}
\begin{aligned}
\boldsymbol{n}^T \cdot {}^{LS}{\boldsymbol{R}} \cdot \boldsymbol{\omega} &= 0 \\
\boldsymbol{n}^T \cdot {}^{LS}{\hat{\boldsymbol{R}}} \left[I - \theta \times  \right]  \cdot \boldsymbol{\omega} &= 0 \\
\boldsymbol{n}^T \cdot {}^{LS}{\hat{\boldsymbol{R}}} \cdot \boldsymbol{\omega} &= \boldsymbol{n}^T \cdot {}^{LS}{\hat{\boldsymbol{R}}} \left[\theta \times  \right]  \cdot \boldsymbol{\omega} \\
\boldsymbol{n}^T \cdot {}^{LS}{\hat{\boldsymbol{R}}} \cdot \boldsymbol{\omega} &= -(\boldsymbol{n}^T \cdot {}^{LS}{\hat{\boldsymbol{R}}} \left[\omega \times  \right])  \cdot \boldsymbol{\theta} \\
\end{aligned}
\end{equation}

It uses the dot product algebra as described in appendix \ref{appendix_dot_product}.


%
\appendix
\section{Linear algebra dot product}\label{appendix_dot_product}

\begin{equation} \label{eq:rekenregel_dot_product}
\begin{aligned}
\boldsymbol{n}^T \cdot (A \cdot \boldsymbol{x}) =(\boldsymbol{n}^T A) \cdot \boldsymbol{x}
\end{aligned}
\end{equation}



%\nocite{*}
\printbibliography



\end{document}
