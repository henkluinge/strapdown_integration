\documentclass{article}

\usepackage{amsmath, amsfonts}
%\usepackage{tensor}
\usepackage[colorlinks]{hyperref}

\usepackage[format=plain,
            labelfont={bf,it},
            textfont=it]{caption}

\usepackage{graphicx, subcaption}
\graphicspath{ {./images/} }

\usepackage{biblatex} %Imports biblatex package
\addbibresource{./strapdown_report.bib} %Import the bibliography file

%\usepackage{biblatex} %Imports biblatex package
%\usepackage[backend=bibtex,style=alphabetic]{biblatex}
%\usepackage[toc]{appendix}  % Show appendices in table of contents.

%\addbibresource{my_library.bib} %Import the bibliography file

\title{Inertial sensor unit error model}
\author{Henk Luinge  \\
	infertia.com   \\
	}
	
\date{\today}

\begin{document}

\maketitle

\begin{abstract}
We describe how the uncertainty of the IMU orientation and velocity estimation propagates over time, given a baseline random walk model.
\end{abstract}

\section{Introduction}

We assume an orientation with some uncertainty, that used to compute a next orientation using an angular velocity/delta angle which has an uncertain baseline. The orientation propagation consists of a deterministic and a stochastic part and can be used as part of a larger maximum likelihood mechanization such as a Kalman filter. We assume consumer-grade inertial sensors as decribed in the strapdown navigation report \cite{Luinge2025}.

\section{Background and definitions}

\paragraph{Lemma: small angles rotation}
We use the helical angle approximation as described in  \cite{Bortz1970} to approximate a rotation from A to B over a small angle $\Delta \varphi$.

\begin{equation} \label{eq:small_angles_orientation_update}
{}^{SB} \varphi = {}^{SA} \varphi + \Delta \varphi + {}^{SA} \varphi \times \Delta \varphi
\end{equation}
This equation is only valid for small angles $\varphi $ as well as $\Delta \varphi$.

The coordinate systems are the same as defined in \cite{Luinge2025}: L stands for the local, earth-fixed coordinate frame and S for the imu frame.




\paragraph{Error definitions}
The orientation error is described using the axis-angle representation $\theta$ and the baseline $b_t$, both 3 element vectors.

\begin{equation} \label{eq:orientation_error_definition}
{}^{LS} \hat{R} = {}^{LS} R \cdot R_{\epsilon} \approx {}^{LS} R \cdot \left[ I + \theta\times  \right]
\end{equation}

The capital $I$ is the 3x3 identity matrix. $\theta$ is the axis and angle to rotate from the real to the estimated orientation, starting from the actual (but unknown) S frame. In typical Kalman filtering applications, the unknown $\theta$ its described using as a zero-mean gaussian probability density function (pdf).

\begin{equation} \label{eq:angular_velocity_model}
y_{gyr,t} =  \hat{\omega}_t = \omega_t + b_t
\end{equation}

\section{Error propagation}
Using \ref{eq:small_angles_orientation_update} and starting from an error axis-angle $\varphi$:
\begin{equation} \label{eq:phi_propagation}
\varphi_t =  \varphi_{t-1} + T\omega + \varphi_{t-1} \times T\omega 
\end{equation}

\section{Strapdown integration}
Assuming constant angular velocity during a sampling interval, the change in orientation between two timesteps is computed as:

\begin{equation} \label{eq:strapdown_orientation_including_bias}
\Delta \hat{R}_t \equiv {}^{S_{t-1}S_t} \hat{R} \approx \left[ I + \omega \times   +  \hat{b}\times  \right]
\end{equation}

or including the previous orientation:
\begin{equation} \label{eq:strapdown_orientation_propagation_rotation_matrix}
\begin{aligned}
{}^{LS} \hat{R}_t &= {}^{LS} \hat{R}_{t-1} \cdot  \Delta \hat{R}_t  \\
&= {}^{LS} R_{t-1} \cdot \left[ I + \theta\times  \right] \cdot \left[ I + \omega \times   +  \hat{b}\times  \right] \\
&=
\end{aligned}
\end{equation}


\printbibliography

\end{document}
